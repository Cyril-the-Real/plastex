\documentclass{article}

\usepackage{amsmath}
\usepackage{amsthm}
\usepackage{graphicx}
\usepackage{tikz}
\usepackage{tikz-cd}

\newtheorem{theorem}{Theorem}
\newtheorem{proposition}[theorem]{Proposition}
\newtheorem{lemma}[theorem]{Lemma}
\newtheorem{remark}[theorem]{Remark}


\DeclareMathOperator{\GL}{GL}
\DeclareMathOperator{\PGL}{PGL}
\DeclareMathOperator{\SL}{SL}
\DeclareMathOperator{\ad}{ad}
\newcommand{\Z}{\mathbf{Z}}
\newcommand{\Gad}{{G_{\ad}}}
\newcommand{\Gm}{\mathbf{G}_m}
\newcommand{\Gtilde}{{\tilde{G}}}
\newcommand{\Ttilde}{{\tilde{T}}}


\title{Sample document}

\author{Patrick Massot}

\begin{document}
\maketitle

\section*{Introduction}

This is a sample \LaTeX document intended to show what plas\TeX can do. It is
made of random excerpts of mathematical texts.


\section{Basic typesetting}

Of course you can type paragraphs containing some mathematics, such as the following one.

If $G=\GL_2$ then $\Gad=\PGL_2$ and so, as above, $G_1=\SL_2$.
Now $G_2$ is the set of $(g,h)\in\SL_2\times\GL_2$ with $g=h$ in $\PGL_2$,
so $h=\lambda g$ for some unique $\lambda\in\Gm$ and $G_2=\SL_2\times\Gm$,
with the obvious map to $\GL_2$ sending $\Gm$ into the centre
(or perhaps its inverse depending on how one is thinking
about things, but this doesn't matter). The subgroup $\mu_2$
is embedded diagonally of course, because it's the kernel
of the map $G_2\to G$. Finally we push out via $\mu_2\to\Gm$
and this gives us $\SL_2\times\Gm\times\Gm$ modulo
the subgroup of order 2 with non-trivial element $(-1,-1,-1)$.
But there's an automorphism of $\Gm\times\Gm$ sending $(-1,-1)$
to $(-1,1)$ (namely, send $(x,y)$ to $(x,xy)$) so again
$\Gtilde$ is just $G\times\Gm$.

You can also use displayed formulas such as:
\[
	\int_{I \times \Sigma} \Phi^*\omega = \int_I\left(\int_{\Phi_t(\Sigma)} \iota_X \omega\right)dt.
\]
and refer to displayed formulas such as Equation~\ref{eq:stokes} below.
\begin{equation}\label{eq:stokes}
  \int_M d\omega = \int_{\partial M} \omega
\end{equation}

Commutative diagrams using \verb+tikz-cd+ are supported as well.

\begin{center}
  \begin{tikzcd}[row sep=1cm]
  	{}   & \ker \pi' = T_mM \times \{0_p\} \ar[d, hookrightarrow] \ar["T_m F_p", dr] &  &  \\
    \ker \pi = T_\sigma\Sigma \ar[r] \ar[r] \ar[r] \ar[r, hookrightarrow] &
  	T_m M \times T_p P \ar["\pi", bend right, swap, rr] \ar["\pi'", d]  \ar["T_\sigma F",
  	r, swap]& T_{F(\sigma)} N \ar["\rho", r, twoheadrightarrow]& \nu_{F(\sigma)} A \\
     & T_p P &  &
  \end{tikzcd}
\end{center}

\section{Theorems and proofs}

You can state and prove results, and refer to them, for instance
Lemma~\ref{lem:splittings} below.

\begin{lemma}
\label{lem:splittings}
Splittings of $0\to\Gm\to\tilde{G}\to G\to 0$ canonically
biject with twisting elements for~$G$.
\end{lemma}

\begin{proof}
To give a splitting is to give a map $\tilde{G}\to\Gm$
such that the composite $\Gm\to\Gtilde\to\Gm$ is the identity;
then the induced map $\Gtilde\to G\times\Gm$ is an injection
with trivial kernel so is an isomorphism for dimension reasons.
If $\chi:\Gtilde\to\Gm$ is such a character then $\chi$ gives
rise to an element of $X^*(\Ttilde)$ which is Galois-stable,
whose image in~$\Z$ is~1, and which pairs to zero with each coroot
(because $\chi$ factors through the maximal torus quotient of $\Gtilde$).
Conversely to give such a character is to give a splitting.
Now one checks that $\theta-\chi$ has image in $\Z$ equal to zero
so gives rise to an element of $X^*(T)$ which is Galois-stable,
and pairs with each simple coroot to~1---but this is precisely
a twisting element for~$G$. Conversely if $t$ is a twisting element
for~$G$ then $\theta-t$ gives a splitting of the exact sequence.
\end{proof}

\section{Enumerations and tables}

You can use lists such as:

\begin{itemize}
  \item $(\Phi\circ \Psi)_*X=\Phi_*\Psi_*X$
  \item $(\varphi\circ \psi)^*\alpha=\psi^*\varphi^*\alpha$
  \item $\varphi^*d\alpha=d\varphi^*\alpha$
  \item $\Phi^*(\mathcal{L}_X\alpha)=\mathcal{L}_{\Phi^{-1}_*X}\Phi^*\alpha$
  \item $\Phi^*(i_X\alpha)=i_{\Phi^{-1}_*X}\Phi^*\alpha$
\end{itemize}

and tables, possibly inside a figure environment such as Figure~\ref{fig:char}.

\begin{figure}[h]
\begin{center}
  \begin{tabular}{c|ccc}
& (1)  & (12) & (123) \\
   \hline
   $\chi_\mathrm{triv}$ &  1 &  1 &  1 \\
   $\chi_\mathrm{sgn}$ & 1 & -1 & 1 \\
   $\chi_\mathrm{std}$ & 2 & 0  & -1 \\
\end{tabular}
\end{center}
\caption{Character table for $S_3$}
\label{fig:char}
\end{figure}
\end{document}
